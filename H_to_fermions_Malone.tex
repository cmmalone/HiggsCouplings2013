\documentclass{beamer}
\usepackage{epstopdf}
\usepackage{amssymb,amsmath}
\usepackage{hyperref}
\usepackage{verbatim}
\definecolor{links}{HTML}{2A1B81} 
\hypersetup{colorlinks,linkcolor=,urlcolor=links}
%\usepackage{xcolor}
\usepackage{pgfplots}
\usepackage{tikz}
%\usepackage[usenames,dvipsnames]{color}
\usetikzlibrary{arrows,shapes,positioning}
\definecolor{BrickRed}{RGB}{178,34,34}
\definecolor{Navy}{RGB}{0,0,128}
\definecolor{Teal}{RGB}{0,128,128}
\definecolor{Green}{RGB}{0,128,0}



\usepackage{graphicx}
\usepackage{multirow}
\usepackage[overlay]{textpos}
\usetheme{Copenhagen}
\usecolortheme{beaver}
\usecolortheme{orchid}
\setbeamertemplate{navigation symbols}{\insertframenumber} 
\colorlet{lightgray}{gray!40}
\setbeamercolor{postit}{fg=purple,bg=lightgray}
% \usepackage{beamerthemesplit} // Activate for custom appearance


\addtobeamertemplate{frametitle}{}{%
\begin{textblock*}{50mm}(.92\textwidth,-0.8cm)
\includegraphics[height=0.55cm,width=1.5cm]{/Users/caitlinmalone/Documents/ATLAS/slac_logo.pdf}
\end{textblock*}}

% \usepackage{beamerthemesplit} // Activate for custom appearance

\title{Higgs Bosons Decaying to Fermions in ATLAS and CMS}
%\subtitle{Higgs Couplings 2013, Freiburg}
\author{Caitlin Malone, SLAC}
\institute{on behalf of the ATLAS Collaboration\\Higgs Couplings 2013, Freiburg\\14 October 2013 }
\date{}

\begin{document}

\frame{\titlepage}



\begin{frame}{Background on this talk (this is not a real slide)}
	\begin{itemize}
		\item Title: ``Higgs boson decays to fermions at ATLAS and CMS''
		\item Conference: Higgs Couplings 2013 (Frieburg)
		\item Time: 30+5 minutes
		\item Audience: 80ish(?) experts, who are interested in the guts of the analyses (according to the organizers)
		\item 3rd talk of the conference, further talks will cover couplings, mass, BSM Higgs sector, etc. in more detail
		\begin{itemize}
			\item Leading talk: ``Electroweak Symmetry Breaking'' (J. Iliopolous)
			\item Second talk: ``Higgs decays to bosons at ATLAS and CMS'' (some CMS person)
			\item Third talk: yours truly
		\end{itemize}
		\item The slides shown today still need lots of formatting work--comments are more solicited on content and high-level stuff than details
	\end{itemize}
\end{frame}



\begin{frame}{Introduction}
	\begin{itemize} \scriptsize
		\item New Higgs boson discovered near 125 GeV first seen and studied in bosonic channels \textcolor{Teal}{($H\rightarrow\gamma\gamma$, $H\rightarrow ZZ\rightarrow 4l$, $H\rightarrow WW \rightarrow l\nu l \nu$)}
		\item At a mass of 125 GeV, Higgs boson decays in many channels, including \textcolor{BrickRed}{quarks and leptons $H\rightarrow bb$, $H\rightarrow\tau\tau$, $H\rightarrow\mu\mu$}
	\end{itemize}
	\begin{columns}[c]
		\column{0.6\textwidth}
			\includegraphics[width=\textwidth]{/Users/caitlinmalone/Documents/ATLAS/HiggsCouplings2013/figures/h_br_sm.pdf}
		\column{0.4\textwidth}
	\end{columns}

\end{frame}


\begin{frame}{Production Channels}
	\begin{columns}[c]
	\column{0.25\textwidth}
		\scriptsize \textcolor{BrickRed}{gluon fusion:} \\
		- $\sigma\approx$ 15 pb \\
		- can get additional jets or high-$p_T$ Higgs \\
		- top quark loop provides main constraint on Higgs couplings to quarks
		\vspace{1cm}
		
		\textcolor{BrickRed}{W/Z associated production:} \\
		- $\sigma\approx$0.58 (W) or 0.34 (Z) pb \\
		- vector boson allows easier triggering and tagging
		
	\column{0.5\textwidth}
		\includegraphics[width=\textwidth]{/Users/caitlinmalone/Documents/ATLAS/HiggsCouplings2013/figures/higgs_feyn_pp.pdf}
		
		
	\column{0.25\textwidth}
		\scriptsize \textcolor{BrickRed}{ttH:}\\
		- $\sigma\approx$ 0.086 pb\\
		- top quarks allow easier triggering and tagging
		
		\vspace{2cm}
		
		\textcolor{BrickRed}{Vector Boson Fusion:} \\
		- $\sigma\approx$1.2 pb \\
		- forward jets provide unique signature		
	\end{columns}
\end{frame}






\begin{frame}[c]
	\frametitle{\ }
	\begin{center}
	\huge \textcolor{Navy}{$H\rightarrow\tau\tau$}
	\end{center}
\end{frame}



\begin{frame}{$H\rightarrow \tau \tau$ Motivation}
	\begin{columns}[c]
	\column{0.6\textwidth}
	\begin{itemize} \scriptsize

		\item \textcolor{BrickRed}{Direct coupling to lepton sector}
		\item 4th channel for observation, after vector bosons and $\gamma \gamma$
		\item \textcolor{BrickRed}{6.3\% branching ratio at $m_H$=125 GeV}
		\item \textcolor{Navy}{Analyses categorized by decay mode, to allow optimization to different backgrounds}

	\end{itemize}
	\column{0.4\textwidth}
		\includegraphics[width=\textwidth]{figures/tau_decay.pdf}
	\end{columns}
	
	\vspace{1cm}

	\begin{columns}
		\column{0.33\textwidth}
			$\tau_{lep}\tau_{lep}$
			\begin{itemize} \scriptsize
				\item lep=$\mu, e$
				\item \textcolor{Navy}{cleanest channel}
				\item $\mu$ generally cleaner than $e$ in detector, lower backgrounds
				\item \textcolor{BrickRed}{price: 12\% branching fraction, Drell-Yan backgrounds}
			\end{itemize}

		\column{0.33\textwidth}
			$\tau_{had}\tau_{had}$
			\begin{itemize} \scriptsize
				\item \textcolor{Navy}{42\% branching ratio, small Drell-Yan background}
				\item \textcolor{BrickRed}{price: larger QCD jets background, jet energy scale/resolution}
			\end{itemize}

		\column{0.33\textwidth}

			$\tau_{lep}\tau_{had}$

				\begin{itemize}	\scriptsize		
					\item \textcolor{Navy}{best of both worlds: clean lepton tag, 46\% branching ratio}
					\item \textcolor{BrickRed}{generally channel with most power}
				\end{itemize}
	\end{columns}
\end{frame}




\begin{frame}{$H \rightarrow \tau \tau$: Overview of Analyses}
	\begin{table}
%		\scriptsize
		\begin{tabular}{c | c | c | c | c | c | c}
		
		
		 & VBF & Boosted & VH & 1-jet$^*$ & ttH & 0-jet$^{**}$\\ \hline \hline
		 
				
			$\mu \tau_{had}$ &
			\includegraphics[width=0.05\textwidth]{figures/atlas_logo.pdf} \includegraphics[width=0.05\textwidth]{figures/cms_logo.pdf} &
			\includegraphics[width=0.05\textwidth]{figures/atlas_logo.pdf} &
			\includegraphics[width=0.05\textwidth]{figures/cms_logo.pdf}&

			\includegraphics[width=0.05\textwidth]{figures/atlas_logo.pdf} \includegraphics[width=0.05\textwidth]{figures/cms_logo.pdf} &
			&
			\includegraphics[width=0.05\textwidth]{figures/atlas_logo.pdf} \includegraphics[width=0.05\textwidth]{figures/cms_logo.pdf} 	
			\\	
			$e \tau_{had}$ &
			\includegraphics[width=0.05\textwidth]{figures/atlas_logo.pdf} \includegraphics[width=0.05\textwidth]{figures/cms_logo.pdf} &
			\includegraphics[width=0.05\textwidth]{figures/atlas_logo.pdf} &
			\includegraphics[width=0.05\textwidth]{figures/cms_logo.pdf} &
			\includegraphics[width=0.05\textwidth]{figures/atlas_logo.pdf} \includegraphics[width=0.05\textwidth]{figures/cms_logo.pdf} &
			&
			\includegraphics[width=0.05\textwidth]{figures/atlas_logo.pdf} \includegraphics[width=0.05\textwidth]{figures/cms_logo.pdf} 	
			\\	
			
			 
			\hline		
			$\mu e$ &
			\includegraphics[width=0.05\textwidth]{figures/atlas_logo.pdf} \includegraphics[width=0.05\textwidth]{figures/cms_logo.pdf} &
			\includegraphics[width=0.05\textwidth]{figures/atlas_logo.pdf} &
			\includegraphics[width=0.05\textwidth]{figures/atlas_logo.pdf} \includegraphics[width=0.05\textwidth]{figures/cms_logo.pdf}&
			\includegraphics[width=0.05\textwidth]{figures/atlas_logo.pdf} \includegraphics[width=0.05\textwidth]{figures/cms_logo.pdf} &
			&
			\includegraphics[width=0.05\textwidth]{figures/atlas_logo.pdf} \includegraphics[width=0.05\textwidth]{figures/cms_logo.pdf} 	
			\\
			$\mu\mu$ &
			\includegraphics[width=0.05\textwidth]{figures/atlas_logo.pdf} \includegraphics[width=0.05\textwidth]{figures/cms_logo.pdf} &
			\includegraphics[width=0.05\textwidth]{figures/atlas_logo.pdf} &
			\includegraphics[width=0.05\textwidth]{figures/atlas_logo.pdf} &
			\includegraphics[width=0.05\textwidth]{figures/atlas_logo.pdf} \includegraphics[width=0.05\textwidth]{figures/cms_logo.pdf} &
			&
			\includegraphics[width=0.05\textwidth]{figures/atlas_logo.pdf} \includegraphics[width=0.05\textwidth]{figures/cms_logo.pdf} 	
			\\
			$ee$ &
			\includegraphics[width=0.05\textwidth]{figures/atlas_logo.pdf} &
			\includegraphics[width=0.05\textwidth]{figures/atlas_logo.pdf} &
			\includegraphics[width=0.05\textwidth]{figures/atlas_logo.pdf} &
			\includegraphics[width=0.05\textwidth]{figures/atlas_logo.pdf} &
			&
			\includegraphics[width=0.05\textwidth]{figures/atlas_logo.pdf} 	
			\\	


			
			\hline
			$\tau_{had} \tau_{had}$ &
			\includegraphics[width=0.05\textwidth]{figures/atlas_logo.pdf} \includegraphics[width=0.05\textwidth]{figures/cms_logo.pdf} &
			\includegraphics[width=0.05\textwidth]{figures/atlas_logo.pdf} \includegraphics[width=0.05\textwidth]{figures/cms_logo.pdf} &	
			\includegraphics[width=0.05\textwidth]{figures/cms_logo.pdf}&
			&\includegraphics[width=0.05\textwidth]{figures/cms_logo.pdf} 
			&  
			\\					
				
		\end{tabular}
	\end{table}
	\scriptsize
	\textcolor{BrickRed}{$^{*}$ CMS splits 1-jet category based on $p_T$ of lepton or $\tau$} \newline
	\textcolor{BrickRed}{$^{**}$ ATLAS: 7 TeV only; CMS: control region only} \\
	NB: Analysis definitions not identical between collaborations!  See notes for more detail
\end{frame}



\begin{frame}{$\nu$ Orientation Cuts and $\tau$ Embedding}
	\begin{columns}[c]
		\column{0.4\textwidth}
		$\nu$ Direction Cuts \\
		\begin{itemize} \scriptsize
			\item $\tau$ from H decay highly boosted and decay products collimated
			\item Place cuts on how $E_T^{miss}$ oriented relative to visible $\tau$ decay products
		\end{itemize}

		
		\column{0.6\textwidth}	
		$\tau$ Embedding
			\begin{itemize} \scriptsize
				\item Pure sample of $Z\rightarrow\tau\tau$ events difficult to get in data
				\item Embedding: get data-driven sample of $Z\rightarrow\mu\mu$ events, replace $\mu$ with simulated $\tau$
				\item Gets Z kinematics, pileup, underlying event from data
			\end{itemize}
	\end{columns}
	
	\begin{columns}[c]
		\column{0.4\textwidth}
			\includegraphics[width=0.9\textwidth]{/Users/caitlinmalone/Documents/ATLAS/HiggsCouplings2013/figures/cms_htautau_collinear_diagram.pdf}				
		\column{0.3\textwidth}
			\includegraphics[width=\textwidth]{/Users/caitlinmalone/Documents/ATLAS/HiggsCouplings2013/figures/atlas_htautau_embedding_before.pdf}
		\column{0.3\textwidth}
			\includegraphics[width=\textwidth]{/Users/caitlinmalone/Documents/ATLAS/HiggsCouplings2013/figures/atlas_htautau_embedding_after.pdf}	
	\end{columns}
\end{frame}



\begin{frame}{$\tau$ Invariant Mass Reconstruction}
	\begin{itemize} \scriptsize
		\item Straightforward assumption of $\nu$ collinear with visible $\tau$ decay products gives an unphysical solution when doing $\tau$ reconstruction, especially when $E_T^{miss}$ and boson parent mass are small
		\item ATLAS: Missing Mass Calculator (MMC) \textcolor{BrickRed}{(Looking for an approved ATLAS plot or mass resolution number)}
		\item CMS: SVFit Algorithm (20\% mass resolution after applying algorithm)
		\item Idea: given the $\tau$ decay modes and the event kinematics, allocate the $E_T^{miss}$ according to the maximization of a likelihood function
	\end{itemize}
	\begin{columns}
		\column{0.5\textwidth}
			\includegraphics[width=\textwidth]{/Users/caitlinmalone/Documents/ATLAS/HiggsCouplings2013/figures/cms_htautau_mass_before.pdf}		
		\column{0.5\textwidth}
			\includegraphics[width=\textwidth]{/Users/caitlinmalone/Documents/ATLAS/HiggsCouplings2013/figures/cms_htautau_mass_after.pdf}	
	\end{columns}
\end{frame}






\begin{frame}{CMS $H\rightarrow \tau \tau$: Results}
	\begin{columns}[c]
		\column{0.5\textwidth}
			\includegraphics[width=0.9\textwidth]{/Users/caitlinmalone/Documents/ATLAS/HiggsCouplings2013/figures/cms_htautau_brazil.pdf}\\
		\column{0.5\textwidth}
			\includegraphics[width=0.9\textwidth]{/Users/caitlinmalone/Documents/ATLAS/HiggsCouplings2013/figures/cms_htautau_p_value.pdf}\\
	\end{columns}	
	\vspace{0.5cm}
	\begin{columns}
		\column{0.5\textwidth}
			\scriptsize 
			At $m_H$=125 GeV, observed (expected) 95\% CL upper limit on cross section is 1.0 (1.63) $\times$ SM (background-only hypothesis)
		\column{0.5\textwidth}
			\scriptsize
			For $m_H$=125 GeV, observed (expected) p-value 2.85 (2.62) and best fit value $\mu$=1.1$\pm$0.4 
	\end{columns}	

\end{frame}



\begin{frame}{CMS: $ttH\rightarrow\tau\tau$ and $VH\rightarrow\tau\tau$}
	

	\begin{columns}[c]
		\column{0.5\textwidth}
			$ttH\rightarrow l\nu qq bb$
			\includegraphics[width=1.26\textwidth]{/Users/caitlinmalone/Documents/ATLAS/HiggsCouplings2013/figures/cms_tthtautau_brazil.pdf}
		\column{0.5\textwidth}
			$VH\rightarrow\tau\tau$
			\includegraphics[width=0.9\textwidth]{/Users/caitlinmalone/Documents/ATLAS/HiggsCouplings2013/figures/cms_vhtautau_brazil.pdf}
	\end{columns}
	
%	\begin{columns}[c]
%		\column{0.5\textwidth}
%		\column{0.5\textwidth}		
%	\end{columns}
\end{frame}



\begin{frame}{ATLAS $H\rightarrow\tau\tau$: Results}
	\begin{columns}[c]
		\column{0.5\textwidth}
			\includegraphics[width=\textwidth]{/Users/caitlinmalone/Documents/ATLAS/HiggsCouplings2013/figures/atlas_htautau_brazil.pdf}
		\column{0.5\textwidth}
			\includegraphics[width=\textwidth]{/Users/caitlinmalone/Documents/ATLAS/HiggsCouplings2013/figures/atlas_htautau_p_value.pdf}
	\end{columns}
	
	\begin{columns}[c]
		\column{0.5\textwidth}
			\scriptsize
			For $m_H$=125 GeV, observed (expected) 95\% CL upper limits on cross section is 1.9 (1.2) $\times$ SM (background-only hypothesis)
		\column{0.5\textwidth}
			\scriptsize
			For $m_H$=125 GeV, observed (expected) p-value 1.1 (1.7)$\sigma$ and best fit value $\mu$=0.7$\pm$0.7
	\end{columns}
\end{frame}




\begin{frame}[c]
	\begin{center}
	\huge \textcolor{Navy}{$H\rightarrow bb$}
	\end{center}
\end{frame}



\begin{frame}{$H\rightarrow bb$ Motivation}
		\textcolor{BrickRed}{- Only experimentally visible decay mode to quarks}\\
		- Main way to probe couplings to down-type quarks \\
		\textcolor{Teal}{- Inclusive production (ggF) impossible to observe because of high QCD background}\\
		\textcolor{Navy}{- Require extra particles or a unique topology}\\
		\vspace{0.2cm}
			\scriptsize{
			\hspace{1cm}	- Forward jets: characteristic of VBF topology\\
			\hspace{1cm}	- Top quark pair: \\
			\hspace{1cm}	- Vector bosons: $Z\rightarrow\nu\nu$, $W\rightarrow l\nu$, $Z\rightarrow ll$\\
			}
		\vspace{0.2cm}
		\normalsize
		\textcolor{BrickRed}{- High branching ratio (about 58\%) means that observation is crucial to constrain the overall Higgs width}

\end{frame}


\begin{frame}{CMS VBF $H\rightarrow bb$}	
	\begin{columns}[c]
%		\column{0.5\textwidth}	%
%			\scriptsize Event topology characterized by large $\Delta\eta$ between non-b-tagged jets	5
%			\includegraphics[width=1.1\textwidth]{/Users/caitlinmalone/Documents/ATLAS/HiggsCouplings2013/figures/cms_vbfhbb_delta_eta.pdf}
		\column{0.33\textwidth}
			\scriptsize ANN discriminant based on event topology and b-tag values
			\includegraphics[width=1.1\textwidth]{/Users/caitlinmalone/Documents/ATLAS/HiggsCouplings2013/figures/cms_vbfhbb_ann_output.pdf}
		\column{0.33\textwidth}
			\scriptsize Highest categories of ANN heavily dominated by VBF production (vs. ggF)
			\includegraphics[width=1.1\textwidth]{/Users/caitlinmalone/Documents/ATLAS/HiggsCouplings2013/figures/cms_vbfhbb_ann_categories.pdf}
		\column{0.33\textwidth}
			\scriptsize Depending on the ANN category, the VBF channel can have up to 47\% ggF Higgs signal \\
			\textcolor{BrickRed}{Yesterday's discovery (gg$\rightarrow$H) is today's measurement and tomorrow's background}
	\end{columns}
	\begin{columns}[c]
		\column{0.5\textwidth}
			\includegraphics[width=\textwidth]{/Users/caitlinmalone/Documents/ATLAS/HiggsCouplings2013/figures/cms_vbfhbb_brazil.pdf}
		\column{0.5\textwidth}
		\scriptsize 

		For $m_H$=125 GeV, 95\% CL upper limits on cross section is 3.6 (3.0) $\times$ SM (background-only hypothesis) and observed signal strength $\mu$=0.7$\pm$1.4 \\
		\textcolor{Navy}{There is a 20-50\% uncertainty on the $gg\rightarrow H$ normalization, which makes a precise measurement of VBF Higgs production difficult}
	\end{columns}
	\scriptsize\textcolor{Teal}{ATLAS VBF analysis still in progress, with result planned for winter 2014}
\end{frame}




\begin{frame}{$ttH,\ H\rightarrow bb$: ATLAS and CMS differences}
	\begin{columns}
		\column{0.5\textwidth}
			ATLAS \\
				\begin{itemize} \scriptsize
					\item Kinematic fit to reconstruct the top quarks, and determine which b quarks to use in Higgs reconstruction
					\item Final discriminant: $m_{bb}$
					\item Includes only $tt\rightarrow l\nu q\bar{q}$
					\item Result: observed (expected) sensitivity of 13.1 (10.5) $\times$ SM
				\end{itemize}
		\column{0.5\textwidth}
			CMS \\
				\begin{itemize} \scriptsize
					\item Result combined with $ttH,\ H\rightarrow\tau\tau$
					\item Final discriminant: output of BDT, look for excess in most Higgs-enriched categories
					\item Includes both $tt\rightarrow l\nu l\nu$ and $tt\rightarrow l\nu q\bar{q}$
				\end{itemize}
	\end{columns}
\end{frame}



\begin{frame}{$ttH,\ H\rightarrow bb$: ATLAS and CMS similarities}
	\begin{columns}[c]
		\column{0.5\textwidth}
			\includegraphics[width=\textwidth]{/Users/caitlinmalone/Documents/ATLAS/HiggsCouplings2013/figures/atlas_tthbb_categories_1.pdf}
		\column{0.5\textwidth}
			\includegraphics[width=\textwidth]{/Users/caitlinmalone/Documents/ATLAS/HiggsCouplings2013/figures/atlas_tthbb_categories_2.pdf}
	\end{columns}
	\begin{itemize} \scriptsize
		\item Binning of analysis in njets, ntags
			\begin{itemize} \scriptsize
				\item ATLAS: 4-6 jets, 0-4 b-tags
				\item CMS varies by channel: 7 combinations ($tt\rightarrow l\nu qq$), 3 combinations ($tt\rightarrow l\nu l\nu$),  6 combinations ($H\rightarrow \tau\tau$)
			\end{itemize}
		\item Also bins analysis in S/B
		\item Recently synchronized systematics on tt+HF background, leading to higher systematics (relative to 2011) for CMS
	\end{itemize}
\end{frame}





\begin{frame}{$ttH,\ H\rightarrow bb$: Results}
	\begin{columns}[c]
		\column{0.5\textwidth}
		ATLAS\\
		\column{0.5\textwidth}
		CMS
	\end{columns}

	\begin{columns}[c]
		\column{0.5\textwidth}
			\includegraphics[width=\textwidth]{/Users/caitlinmalone/Documents/ATLAS/HiggsCouplings2013/figures/atlas_tthbb_brazil.pdf} \\
			\vspace{2.5cm}
					\scriptsize
					top row: ATLAS and CMS results for $ttH\rightarrow l\nu qqbb$\\
					bottom: CMS results for $ttH\rightarrow l\nu l\nu bb$\\
		\column{0.5\textwidth}
			\includegraphics[width=\textwidth]{/Users/caitlinmalone/Documents/ATLAS/HiggsCouplings2013/figures/cms_tthbb_brazil_lepton_jets.pdf}	\\
			\includegraphics[width=\textwidth]{/Users/caitlinmalone/Documents/ATLAS/HiggsCouplings2013/figures/cms_tthbb_brazil_dilepton.pdf}		
	\end{columns}
\end{frame}




\begin{frame}{$VH\rightarrow bb$ at ATLAS and CMS: Similarities}
	\begin{itemize} \scriptsize
		\item $Z\rightarrow \nu\nu$, $W\rightarrow l\nu$, $Z\rightarrow ll$ analyses 
		\item Analyses binned in $p^V_T$ to extract extra sensitivity from most powerful regions
			\begin{itemize} \scriptsize
				\item Trigger on associated vector boson: higher $p_T^V$ means higher efficiency
				\item High $p_T^V$ means harder Higgs, with better resolution of jets (and, by extension, $m_{bb}$)
			\end{itemize}
		\item Jet correction based on $H\rightarrow bb$ used to improve mass resolution
			\begin{itemize} \scriptsize
				\item CMS: improves mass resolution by about 15\%, overall sensitivity by 10-20\%
				\item ATLAS: improves mass resolution by about 10-12\%
				\item \textcolor{BrickRed}{currently tracking down plot or two to show here}
			\end{itemize}
		\item Validated using $Z\rightarrow bb$ events
			\begin{itemize} \scriptsize
				\item CMS: 7.5$\sigma$, $\mu=1.19^{+0.28}_{-0.23}$
				\item ATLAS: 4.8$\sigma$, $\mu=0.9\pm0.2$
			\end{itemize}
	\end{itemize}
	\scriptsize\textcolor{Teal}{Tevatron results help round out the picture} \href{http://arxiv.org/abs/1305.1530}{(link)}
\end{frame}



\begin{frame}{$VH\rightarrow bb$ at ATLAS and CMS: Differences}
	\begin{columns}
		\column{0.5\textwidth}
		ATLAS
			\begin{itemize} \scriptsize
				\item Cut-based analysis 
				\item Signal region has 2 b-tags applied, but use 1-tag and 0-tag regions for background control and validation
				\item 2-tag signal region further subdivided into 2-jet and 3-jet bins
				\item $P_T^V$-based reweighting of important backgrounds (W+jets, t$\bar t$) found to be mismodelled in MC
			\end{itemize}
			\begin{columns}[c]
				\column{0.5\textwidth}
					\includegraphics[width=1.2\textwidth]{/Users/caitlinmalone/Documents/ATLAS/HiggsCouplings2013/figures/atlas_vhbb_delta_phi_uncorrected.pdf}
				\column{0.5\textwidth}
					\includegraphics[width=1.2\textwidth]{/Users/caitlinmalone/Documents/ATLAS/HiggsCouplings2013/figures/atlas_vhbb_delta_phi_corrected.pdf}
			\end{columns}			
		\column{0.5\textwidth}
		CMS
			\begin{itemize} \scriptsize
				\item B-tagging in trigger allows lower $E_T^{miss}$ thresholds for $Z\rightarrow\nu\nu$ analysis
				\item BDTs trained to distinguish 4 main classes of events: $t\bar{t}$, diboson, Z+jets, VH
				\item Final discriminating variable is the BDT output--look for excess of event relative to background in the most signal-enriched BDT bins
				\item Validation cut-based analysis based on $m_{bb}$ \textcolor{BrickRed}{currently looking into whether there is a sensitivity computed for this analysis}
			\end{itemize}
			\includegraphics[width=0.7\textwidth]{/Users/caitlinmalone/Documents/ATLAS/HiggsCouplings2013/figures/cms_vhbb_mbb_distribution.pdf}			
	\end{columns}
\end{frame}





\begin{frame}{CMS $VH (H\rightarrow bb)$}
	\scriptsize
	 \textcolor{Teal}{BDT-based analysis}, where separate BDT trained for each channel W($l\nu$)H, W($\tau\nu$)H, Z($ll$)H, Z($\nu\nu$)H
	 
	 \vspace{0.5cm}
	 
	 \begin{columns}[c]
	 	\column{0.5\textwidth}
			\includegraphics[width=0.8\textwidth]{/Users/caitlinmalone/Documents/ATLAS/HiggsCouplings2013/figures/cms_vhbb_bdt_sensitive.pdf}	\\
	
		\column{0.5\textwidth}
			\includegraphics[width=0.8\textwidth]{/Users/caitlinmalone/Documents/ATLAS/HiggsCouplings2013/figures/cms_vhbb_bdt_all.pdf}\\

	\end{columns}
	
	\vspace{0.5cm}
	
	\begin{columns}
		\column{0.5\textwidth}
			\scriptsize
			Example output of the BDT, focusing on the \textcolor{BrickRed}{most signal-enriched component of the high-$p_T$ Z($\nu\nu$) bin}
		\column{0.5\textwidth}
			\scriptsize
			\textcolor{Navy}{Combination of all BDT discriminants.  The two bottom insets how the ratio of the data to the background-only prediction (above) and to the predicted sum of signal plus background (below).}
	\end{columns}
\end{frame}





\begin{frame}{$VH (H\rightarrow bb)$: CMS Results}
	\begin{columns}[c]
		\column{0.5\textwidth}
			\includegraphics[width=\textwidth]{/Users/caitlinmalone/Documents/ATLAS/HiggsCouplings2013/figures/cms_vhbb_brazil.pdf}
		\column{0.5\textwidth}
			\includegraphics[width=\textwidth]{/Users/caitlinmalone/Documents/ATLAS/HiggsCouplings2013/figures/cms_vhbb_p_value.pdf}
	\end{columns}
		
	\begin{columns}[c]
		\column{0.5\textwidth}
			\scriptsize
			For $m_H$=125 GeV, observed (expected) 95\% CL upper limits on cross section is 1.89 (0.95) $\times$ SM (background-only hypothesis)
		\column{0.5\textwidth}
			\scriptsize
			For $m_H$=125 GeV, BDT has observed p-value 2.1$\sigma$ and best fit value $\mu$=1.0$\pm$0.5
	\end{columns}
\end{frame}






\begin{frame}{ATLAS $VH (H\rightarrow bb)$: Strategy}
	\scriptsize
	\textcolor{Green}{Each $p_T^V$ category in ATLAS further divided into 2-jet and 3-jet signal regions}\\
	2-jet signal region has S/B about 2$\times$ higher than 3-jet region for all categories\\
	\textcolor{Green}{Below are the $m_{bb}$ distributions from the 2-jet, 2-tag, $p_T^V>$200 GeV regions (which are the most signal-enriched)}
	\vspace{-0.2cm}

	\begin{columns}[c]
	
		\column{0.33\textwidth}
			\begin{center} \scriptsize $Z\rightarrow\nu\nu$ \\ \end{center} 
			\vspace{-0.1cm}
			\tiny{0 leptons\\
			\textcolor{Green}{2 b-tags, $p_T^{jet1}>$45 GeV, $p_T^{jet2}>$20 GeV\\}
			$+\le$ 1 extra jets\\
			\textcolor{Green}{$E^{miss}_{T}$ and $p_T^{miss}$ cuts to minimize dijet QCD}
			}
			\includegraphics[width=\textwidth]{/Users/caitlinmalone/Documents/ATLAS/HiggsCouplings2013/figures/atlas_vhbb_0lep_2jet_2tags_200GeV.pdf}\\
			
			
		\column{0.33\textwidth}
			\begin{center} \scriptsize $W\rightarrow l\nu$ \\ \end{center} 
			\vspace{-0.1cm}
			\tiny{1 lepton\\
			\textcolor{Green}{2 b-tags, $p_T^{jet1}>$45 GeV, $p_T^{jet2}>$20 GeV\\}
			$+\le$ 1 extra jets\\
			\textcolor{Green}{$E^{miss}_{T}>$ 25 GeV\\}
			$m_T^W<$120 GeV
			}
			\includegraphics[width=\textwidth]{/Users/caitlinmalone/Documents/ATLAS/HiggsCouplings2013/figures/atlas_vhbb_1lep_2jet_2tags_200GeV.pdf}\\
			
		\column{0.33\textwidth}
			\begin{center} \scriptsize $Z\rightarrow ll$ \\ \end{center} 
			\vspace{-0.1cm}
			\tiny{2 leptons\\
			\textcolor{Green}{2 b-tags, $p_T^{jet1}>$45 GeV, $p_T^{jet2}>$20 GeV\\}
			$+\le$ 1 extra jets\\
			\textcolor{Green}{$E^{miss}_{T}<$ 60 GeV\\}
			83$<m_{ll}<$99 GeV
			}		
			\includegraphics[width=\textwidth]{/Users/caitlinmalone/Documents/ATLAS/HiggsCouplings2013/figures/atlas_vhbb_2lep_2jet_2tags_200GeV.pdf}\\
			
	\end{columns}
	\scriptsize\textcolor{Navy}{ATLAS is also actively investigating multivariate approaches for the $VH\rightarrow bb$ analysis, for possible inclusion in future results}
\end{frame}



\begin{frame}{ATLAS $VH (H\rightarrow bb)$: Results}
	\begin{columns}
		\column{0.5\textwidth}
			\includegraphics[width=\textwidth]{/Users/caitlinmalone/Documents/ATLAS/HiggsCouplings2013/figures/atlas_vhbb_brazil.pdf}
		\column{0.5\textwidth}
			\includegraphics[width=\textwidth]{/Users/caitlinmalone/Documents/ATLAS/HiggsCouplings2013/figures/atlas_vhbb_p_value.pdf}
	\end{columns}
		
	\begin{columns}[c]
		\column{0.5\textwidth}
			\scriptsize
			For $m_H$=125 GeV, observed (expected) 95\% CL upper limits on cross section is 1.4 (1.3) $\times$ SM (background-only hypothesis)
		\column{0.5\textwidth}
			\scriptsize
			For $m_H$=125 GeV, observed (expected) p-value is 0.36 (0.05) and best fit value $\mu=0.2\pm0.5(stat)\pm0.4(syst)$
	\end{columns}
\end{frame}





\begin{frame}[c]
	\begin{center}
	\huge \textcolor{Navy}{$H\rightarrow \mu\mu$}
	\end{center}
\end{frame}




% http://cds.cern.ch/record/1523695/files/ATLAS-CONF-2013-010.pdf
\begin{frame}{$H\rightarrow\mu \mu$ Motivation}
		\begin{itemize} \scriptsize
			\item Small cross section
			\item Clean final state signature
			\item Only channel for measuring coupling to second-generation fermions
			\item Large irreducible background of $Z/\gamma^*\rightarrow\mu\mu$
			\item Can have enhanced BF from non-SM contributions
		\end{itemize}
			\includegraphics[width=0.9\textwidth]{/Users/caitlinmalone/Documents/ATLAS/HiggsCouplings2013/figures/hmumu_theory.pdf}\\
			\scriptsize
			Plots from Tao Han and Bob McElrath showing MSSM enhancements to $H\rightarrow\mu\mu$ as a function of $m_h$ and tan$\beta$ (\href{http://arxiv.org/abs/hep-ph/0201023}{arXiv hep-ph 0201023})
\end{frame}





\begin{frame}{$H\rightarrow\mu \mu$ at ATLAS}
	\begin{columns}[c]
	\column{2.5in}
		\begin{itemize}  \scriptsize
			\item \textcolor{BrickRed}{Reconstruct invariant mass of 2 muons, $p_{T}^{\mu_1}>25$ GeV and $p_{T}^{\mu_2}>15$ GeV}
			\item Remove 60\% of Drell-Yan background events (and keeping 80\% of signal) by requiring p$_T^{\mu+\mu-}>$ 15 GeV (events failing this cut go into a background control region)
			\item \textcolor{Navy}{Search for bump in the invariant mass spectrum, main background is Z+jets}
			\item Background model: exponential plus Breit-Wigner, to capture Z tail
		\end{itemize}
	\column{2.5in}
			\includegraphics[width=0.7\textwidth]{/Users/caitlinmalone/Documents/ATLAS/HiggsCouplings2013/figures/atlas_hmumu_mass_linear.pdf} \\
			
	\end{columns}
	\begin{columns}[c]
		\column{0.25\textwidth}
			\includegraphics[width=1.1\textwidth]{/Users/caitlinmalone/Documents/ATLAS/HiggsCouplings2013/figures/atlas_hmumu_mass_sim_central.pdf}
		\column{0.25\textwidth}
			\includegraphics[width=1.1\textwidth]{/Users/caitlinmalone/Documents/ATLAS/HiggsCouplings2013/figures/atlas_hmumu_mass_data_central.pdf}
		\column{0.25\textwidth}
			\includegraphics[width=1.1\textwidth]{/Users/caitlinmalone/Documents/ATLAS/HiggsCouplings2013/figures/atlas_hmumu_mass_sim_non_central.pdf}
		\column{0.25\textwidth}
			\includegraphics[width=1.1\textwidth]{/Users/caitlinmalone/Documents/ATLAS/HiggsCouplings2013/figures/atlas_hmumu_mass_data_non_central.pdf}
	\end{columns}
	\begin{columns}[c]
		\column{0.5\textwidth}
		\tiny{\textcolor{BrickRed}{Simulation and data in central region ($|\eta(\mu_{1,2})|<1.0$), fit with BW + exponential}}
		\column{0.5\textwidth}
		\tiny{\textcolor{BrickRed}{Simulation and data in non-central region ($|\eta(\mu_{1,2})|>1.0$), fit with BW + exponential}}
	\end{columns}
	
\end{frame}


\begin{frame}{$H\rightarrow\mu\mu$ Results at ATLAS}
	\begin{columns}
		\column{2.5in}
			\includegraphics[width=0.9\textwidth]{/Users/caitlinmalone/Documents/ATLAS/HiggsCouplings2013/figures/atlas_hmumu_brazil.pdf}
		\column{2.5in}
			\includegraphics[width=0.9\textwidth]{/Users/caitlinmalone/Documents/ATLAS/HiggsCouplings2013/figures/atlas_hmumu_p_value.pdf}			
	\end{columns}
	
	\begin{table}
	\scriptsize
	\begin{tabular}{c | c | c | c | c | c | c } 
	\hline 
	$m_H$ & observed limits & exp. median & exp. + 2$\sigma$ & exp. +1$\sigma$ & exp. -1$\sigma$ & exp. -2$\sigma$ \\ \hline
	110 & 5.1 & 10.4 & 20.0 & 14.6 & 7.5 & 5.6 \\
	115 & 5.7 & 7.5 & 14.5 & 10.6 & 5.4 & 4.0 \\
	120 & 9.2 & 7.6 & 14.6 & 10.7 & 5.5 & 4.1 \\
	\textcolor{BrickRed}{125} & \textcolor{BrickRed}{9.8} &\textcolor{BrickRed}{8.2} & \textcolor{BrickRed}{15.9} & \textcolor{BrickRed}{11.6} & \textcolor{BrickRed}{5.9} & \textcolor{BrickRed}{4.4} \\
	130 & 10.8 & 9.1 & 17.5 & 12.8 & 6.5 & 4.9 \\
	135 & 11.0 & 10.4 & 20.1 & 14.6 & 7.5 & 5.6 \\
	140 & 16.8 & 12.9 & 25.0 & 18.2 & 9.3 & 6.9 \\
	145 & 16.9 & 18.3 & 35.3 & 25.7 & 13.2 & 9.8 \\ \hline
	\end{tabular}
	\end{table}
	\scriptsize\textcolor{Navy}{CMS analysis underway, with results planned for fall 2014}
\end{frame}


\begin{frame}{$H\rightarrow\mu\mu$ Results at CMS}
to be included if approved
\end{frame}


\begin{frame}{Summary of Channels}
	\begin{columns}[c]
		\column{0.5\textwidth}
			ATLAS \\
			\includegraphics[width=0.9\textwidth]{/Users/caitlinmalone/Documents/ATLAS/HiggsCouplings2013/figures/atlas_mu_all_channels.pdf}
		\column{0.5\textwidth}
			CMS\\
			\includegraphics[width=0.9\textwidth]{/Users/caitlinmalone/Documents/ATLAS/HiggsCouplings2013/figures/cms_mu_all_channels.pdf}		\end{columns}
	\begin{columns}[c]
		\column{0.5\textwidth}

			\scriptsize
			$VH\rightarrow Vbb$ coming into more consistency with SM $\mu$ expectation  \\
			\textcolor{BrickRed}{Global best-fit signal strength: 1.30$\pm$0.13(stat)$\pm$0.14(sys) at 125.5 GeV}
		\column{0.5\textwidth}			
			\scriptsize
			$VH\rightarrow bb$ slightly above $\mu=1$\\
			\textcolor{BrickRed}{Global best-fit signal strength: 0.80$\pm$0.14 at 125.7 GeV}
		
	\end{columns} 
	\vspace{0.5cm}
	\scriptsize
	$H\rightarrow bb$ has largest error bar, so least effect on final fit value for $\mu$\\
	$H\rightarrow \tau \tau$ consistent with SM for both collaborations
\end{frame}


\begin{frame}{Implications of Fermionic Channels for Understanding the New Boson}
	\begin{itemize} \scriptsize
		\item ATLAS Global fit of $\mu=1.3\pm0.13(stat)\pm0.14(syst)$ has p-value showing 9\% consistency with SM
		\item However, when you vary QCD scale and PDF for the ggF production, consistency of observed signal strength with SM increases to about 40\%
		\item Now that bb is on the scene, any first fits of the total width?
	\end{itemize}
\end{frame}



\begin{frame}{Prospects for 2015 and beyond}

\end{frame}


\begin{frame}
	\textcolor{Navy}{Additional Information}
\end{frame}


\begin{frame}{References}
	\begin{itemize} \scriptsize
		\item ATLAS, \href{https://atlas.web.cern.ch/Atlas/GROUPS/PHYSICS/CONFNOTES/ATLAS-CONF-2013-079/}{ \textit{Search for the bb decay of the Standard Model Higgs boson in associated (W/Z)H production with the ATLAS detector}}, 19 July 2013
		\item ATLAS, \href{https://atlas.web.cern.ch/Atlas/GROUPS/PHYSICS/CONFNOTES/ATLAS-CONF-2012-135/}{\textit{Search for the Standard Model Higgs boson produced in association with top quarks in proton-proton collisions at $\sqrt s$=7 TeV using the ATLAS detector}}, 15 September 2012
		\item CMS, \href{http://cds.cern.ch/record/1546801?ln=en}{\textit{Search for the SM Higgs boson produced in association with W or Z bosons, and decaying to bottom quarks}}, 14 May 2013
		\item CMS, \href{http://cds.cern.ch/record/1564682?ln=en}{\textit{Search for Higgs Boson Production in association with a top-quark pair and decaying to bottom quarks or tau leptons}}, 26 July 2013
		\item ATLAS, \href{https://atlas.web.cern.ch/Atlas/GROUPS/PHYSICS/CONFNOTES/ATLAS-CONF-2012-160/}{\textit{Search for the Standard Model Higgs boson in H to tau tau decays in proton-proton collisions with the ATLAS detector}}, 13 November 2012
		\item CMS, \href{https://cds.cern.ch/record/1528271?ln=en}{\textit{Search for the standard model Higgs boson decaying to tau pairs in proton-proton collisions at $\sqrt s$=7 and 8 TeV}}, 15 March 2013
		\item CMS, \href{https://cds.cern.ch/record/1528147/files/HIG-12-053-pas.pdf}{\textit{Search for the standard model Higgs boson decaying to tau pairs produced in association with a W or Z boson with the CMS experiment in pp collisions at $\sqrt s$ = 7 and 8 TeV}}
		\item ATLAS, \href{https://atlas.web.cern.ch/Atlas/GROUPS/PHYSICS/CONFNOTES/ATLAS-CONF-2013-010/}{\textit{Search for the SM Higgs boson in H to mu mu decays with the ATLAS detector}}, 5 March 2013
		\item LHC Higgs Cross Section Working Group, \href{Handbook of LHC Higgs Cross Sections: 1. Inclusive Observables}{\textit{http://arxiv.org/abs/1101.0593}}, 20 May 2011
		\item LHC Higgs Cross Section Working Group, \href{http://arxiv.org/abs/1201.3084}{\textit{Handbook of LHC Higgs Cross Sections: 2. Differential Distributions}}
	\end{itemize}
\end{frame}




\begin{frame}{$H \rightarrow \tau\tau$: $\tau_{lep}\tau_{lep}$ VBF}
	\begin{columns}[c]
		\column{0.4\textwidth}
			\includegraphics[width=0.92\textwidth]{figures/cms_htautau_mumu_VBF.pdf} \\	
		\column{0.2\textwidth} \scriptsize
			CMS breaks down results by final state of $\tau$`s ($\mu \mu$ vs. $e\mu$) \\ 
			\textcolor{BrickRed}{7 TeV and 8 TeV data combined}
		\column{0.4\textwidth}	
			\includegraphics[width=0.92\textwidth]{figures/cms_htautau_emu_VBF.pdf}	
	\end{columns}
	
	\begin{center}
	\line(1,0){250}
	\end{center}
	
	\begin{columns}[c]
		\column{0.4\textwidth}
			\includegraphics[width=0.92\textwidth]{figures/atlas_htautau_leplep_VBF_7TeV.pdf}		
		\column{0.2\textwidth} \scriptsize
				ATLAS breaks down results by energy (7 TeV vs. 8 TeV) \\
				\textcolor{BrickRed}{$\tau$ final states combined ($ee$, $e\mu$, $\mu\mu$)}
		\column{0.4\textwidth}
			\includegraphics[width=0.92\textwidth]{figures/atlas_htautau_leplep_VBF_8TeV.pdf} 
	\end{columns}
\end{frame}



\begin{frame}{$H\rightarrow \tau \tau$: CMS Channel Breakdown}
	\includegraphics[width=\textwidth]{/Users/caitlinmalone/Documents/ATLAS/HiggsCouplings2013/figures/cms_htautau_signal_strength.pdf}
\end{frame}



\begin{frame}{$H\rightarrow\tau\tau$: ATLAS Channel Breakdown}
	\begin{columns}[c]
		\column{0.33\textwidth}
			\includegraphics[width=0.9\textwidth]{/Users/caitlinmalone/Documents/ATLAS/HiggsCouplings2013/figures/atlas_htautau_leplep_brazil.pdf}\\
			\scriptsize\center \vspace{-0.5cm}
			$\tau_{lep}\tau_{lep}$
		\column{0.33\textwidth}
			\includegraphics[width=0.9\textwidth]{/Users/caitlinmalone/Documents/ATLAS/HiggsCouplings2013/figures/atlas_htautau_lephad_brazil.pdf} 
			\scriptsize\center \vspace{-0.5cm}
			$\tau_{lep}\tau_{had}$
		\column{0.33\textwidth}
			\includegraphics[width=0.9\textwidth]{/Users/caitlinmalone/Documents/ATLAS/HiggsCouplings2013/figures/atlas_htautau_hadhad_brazil.pdf} 
			\scriptsize \center \vspace{-0.5cm}
			$\tau_{had}\tau_{had}$
	\end{columns}
	
	\begin{center}
	\line(1,0){250}
	\end{center}
	
	\begin{columns}[c]
		\column{0.5\textwidth}
%			\includegraphics[width=0.7\textwidth]{/Users/caitlinmalone/Documents/ATLAS/HiggsCouplings2013/figures/atlas_htautau__brazil.pdf}\\
			\scriptsize \center \vspace{-0.5cm}
			VBF channels
		\column{0.5\textwidth}
%			\includegraphics[width=0.7\textwidth]{/Users/caitlinmalone/Documents/ATLAS/HiggsCouplings2013/figures/atlas_htautau_lephad_brazil.pdf} \\
			\scriptsize \center \vspace{-0.5cm}
			non-VBF channels
	\end{columns}

\end{frame}




\begin{frame}{Binning in $P^V_T$}
	\scriptsize
	\textcolor{Teal}{Backgrounds are substantially reduced by requiring a significant boost of the $p_T$ of the vector boson, $p^V_T$. \\}
	\textcolor{Navy}{The boost categories (all numbers in GeV) below are for CMS and ATLAS.}
	\begin{columns}[c]
		\column{0.5\textwidth}
			\begin{table}
			\scriptsize
			\begin{tabular}{c | c | c | c }
				& low & medium & high \\ \hline
			W($l\nu$) & 100-130 & 130-180 & $>$180 \\
			W($\tau\nu$) & & & $>$120 \\
			Z($\nu\nu$)  & 100-130 & 130-170 & $>$170 \\
			Z($ll$)   & 50-100 & & $>$100 \\
			\end{tabular}
			\end{table}
		\column{0.5\textwidth}
	\includegraphics[width=0.25\textwidth]{/Users/caitlinmalone/Documents/ATLAS/HiggsCouplings2013/figures/cms_logo.pdf}
	\end{columns}

	\begin{columns}[c]
		\column{0.6\textwidth}
			\begin{table}
			\scriptsize
				\begin{tabular}{c | c | c | c | c | c}
				 & low & med-low & medium & med-high & high \\ \hline
				W($l\nu$) & 0-90 & 90-120 & 120-160 & 160-200 & $>$200 \\
				Z($\nu\nu$) & $^{*}$ & $^{*}$ & 120-160 & 160-200 & $>$200 \\
				Z($ll$) & 0-90 & 90-120 & 120-160 & 160-200 & $>$200 \\
				\end{tabular}
			\end{table}
			\tiny{ 
			$^{*}$ $E_{T}^{miss}$ trigger becomes 90\% efficient at $E_{T}^{miss}$=120 GeV }\\
			\vspace{0.3cm}
			\textcolor{BrickRed}{\scriptsize{Each vector boson final state and $p_T$ category is further subdivided into 2-jet and 3-jet signal regions}}
		\column{0.2\textwidth}
	\includegraphics[width=0.6\textwidth]{/Users/caitlinmalone/Documents/ATLAS/HiggsCouplings2013/figures/atlas_logo.pdf}		
	\end{columns}
\end{frame}



\end{document}
